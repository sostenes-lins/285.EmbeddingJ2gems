

\begin{abstract}
In this final part of a 3-part paper we introduce
the pair of ``wings'' of the abstract 
PL-colored complexes $\mathcal{H}_{m}^\star$, described in the second paper.
The wings, via a weight enhanced Tutte's barycentric 
embedding of a planar map, produce the unexpected reformutation of a 3-dimensionl problem
into a 2-dimensional one. The total number of edges in 
each one of the pair of final wings is less than $8n-5$. Tutte's method is applied $O(n)$ times
to each one of the 2 wings in the final pair to assure rectilinearity of the embeddings of the planar maps,
which include the final wings. A cone construction over the final wings provides a PL-complex $\mathcal{H}_1^\diamond$,
which contain the set of 0-simplices $\{a_1, a_2,\ldots,a_f\} \cup \{b_1, b_2,\ldots,b_g\}$ (as defined in the 
second part of the article) properly fixed in $\mathbb{R}^3$.
The other 0-simplices are obtained by bisections of segments linking previously defined points.
This implies that $\mathcal{H}_n$ is PL-embedded into $\mathbb{R}^3$. 
We then conclude the surgery description of
the 3-manifold induced by the gem with its resolution by 
defining some disjoint cylinders contained in  $\mathcal{H}_{n}^\star$,
directly from the hinges (dual of the twistors of the resolution), in a 1-1 correspondence. 
The medial curves of the cylinders define the link we seek. The framing of a medial curve is
the linking number of the boundary components of the corresponding cylinder.
The analysis of the whole proccess
shows that the memory and time requirement to complete the algorithm is
$O(n^2)$. Data for the Weber-Seifert 3-manifold, which answers Jeffrey Weeks's question
is given in the appendix. It consists of a link with 142 crossings but it admits simplifications. 
\end{abstract}

