

\section{Introduction}
A {\em $J^ 2$-gem} is a 4-regular, 4-edge-colored planar graph $G$ obtained from the
intersection pattern of two Jordan curves $X$ and $Y$ with $2n$ transversal
crossings.
These crossings define consecutive segments of $X$ alternatively 
inside $Y$ and outside $Y$. Color the first type 2 and the second type 3.
The crossings also define consecutive segments of $Y$ alternatively 
inside $X$ and outside $X$. Color the first type 0 and the second type 1.
This defines a 4-regular 4-edge-colored graph $G$ where the vertices are the crossings
and the edges are the colored colored segments.
Let $K$ be the 3-dimensional abstract 3-complex formed by taking a set of 
vertex colored tetrahedra in 1--1 correspondence with $V(G)$ so that each tetrahedra
has vertices of colors 0,1,2,3. For each $i$-colored edge of $G$ with ends $u$ and $v$
paste the corresponding tetrahedra $\nabla_u$ and $\nabla_v$ so as to paste the two triangular
faces that do not contain a vertex of color $i$ in such a way as to
to match vertices of the other three colors. 
We show that the topological space 
$|K|$ induced by $K$ is $\mathbb{S}^ 3$. Moreover we describe an $O(n^2)$-algorithm
to make available a PL-embedding (\cite{rourke1982introduction}) of $G^\star$ 
into $\mathbb{S}^3$. We get explicit coordinates in $\mathbb{S}^ 3$ for the 0-simplices and
the p-simplices $(p \in \{1,2,3\}$) are linear simplices in the spherical geometry.


